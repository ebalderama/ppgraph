\documentclass{statsoc}
\usepackage[]{graphicx}
\usepackage[]{color}

%% maxwidth is the original width if it is less than linewidth
%% otherwise use linewidth (to make sure the graphics do not exceed the margin)
\makeatletter
\def\maxwidth{ %
  \ifdim\Gin@nat@width>\linewidth
    \linewidth
  \else
    \Gin@nat@width
  \fi
}
\makeatother

\definecolor{fgcolor}{rgb}{0.345, 0.345, 0.345}
\newcommand{\hlnum}[1]{\textcolor[rgb]{0.686,0.059,0.569}{#1}}%
\newcommand{\hlstr}[1]{\textcolor[rgb]{0.192,0.494,0.8}{#1}}%
\newcommand{\hlcom}[1]{\textcolor[rgb]{0.678,0.584,0.686}{\textit{#1}}}%
\newcommand{\hlopt}[1]{\textcolor[rgb]{0,0,0}{#1}}%
\newcommand{\hlstd}[1]{\textcolor[rgb]{0.345,0.345,0.345}{#1}}%
\newcommand{\hlkwa}[1]{\textcolor[rgb]{0.161,0.373,0.58}{\textbf{#1}}}%
\newcommand{\hlkwb}[1]{\textcolor[rgb]{0.69,0.353,0.396}{#1}}%
\newcommand{\hlkwc}[1]{\textcolor[rgb]{0.333,0.667,0.333}{#1}}%
\newcommand{\hlkwd}[1]{\textcolor[rgb]{0.737,0.353,0.396}{\textbf{#1}}}%
\let\hlipl\hlkwb

\usepackage{framed}
\makeatletter
\newenvironment{kframe}{%
 \def\at@end@of@kframe{}%
 \ifinner\ifhmode%
  \def\at@end@of@kframe{\end{minipage}}%
  \begin{minipage}{\columnwidth}%
 \fi\fi%
 \def\FrameCommand##1{\hskip\@totalleftmargin \hskip-\fboxsep
 \colorbox{shadecolor}{##1}\hskip-\fboxsep
     % There is no \\@totalrightmargin, so:
     \hskip-\linewidth \hskip-\@totalleftmargin \hskip\columnwidth}%
 \MakeFramed {\advance\hsize-\width
   \@totalleftmargin\z@ \linewidth\hsize
   \@setminipage}}%
 {\par\unskip\endMakeFramed%
 \at@end@of@kframe}
\makeatother

\definecolor{shadecolor}{rgb}{.97, .97, .97}
\definecolor{messagecolor}{rgb}{0, 0, 0}
\definecolor{warningcolor}{rgb}{1, 0, 1}
\definecolor{errorcolor}{rgb}{1, 0, 0}
\newenvironment{knitrout}{}{} % an empty environment to be redefined in TeX

\usepackage{alltt}
\usepackage{amsfonts,epsfig,eurosym, amsmath, amsthm,  blindtext, enumitem}
\usepackage[latin1]{inputenc}
\usepackage[T1]{fontenc}
\usepackage{pdfsync}
\usepackage{natbib}
\usepackage{rotating}
\usepackage[margin=40pt,font=small,labelfont=bf,format=plain,indention=0cm,textfont=it]{caption}
\usepackage[normalem]{ulem}
\usepackage{bbm}
\usepackage{url}


%======================================================%
%-----------------------------------------------------------------------------------------------%
%****************************************************************************%
%++++++++++++++++++++++++++++++++++++++++++++++++++++++%
%~~~~~~~~~~~~~~~~~~~~~~~~~~~~~~~~~~~~~~~~~~~~~~~~~~~~~~%
%  My Packages and Commands %

%\usepackage{fullpage}
\usepackage{setspace}
\usepackage[left=1in,top=1in,right=1in]{geometry}
\pdfpagewidth 8.5in
\pdfpageheight 11in 
\setlength{\textheight}{9in}


% BIBLIOGRAPHY
\usepackage[authoryear]{natbib}
\bibpunct{(}{)}{;}{a}{}{,}
%\linespread{1.5}

\long\def\symbolfootnote[#1]#2{\begingroup%
\def\thefootnote{\fnsymbol{footnote}}\footnote[#1]{#2}\endgroup}



%============================================
%============================================
%My New Commands
%============================================
\newcommand{\balpha}{\mbox{\boldmath $\alpha$} }
\newcommand{\bbeta}{\mbox{\boldmath $\beta$} }
\newcommand{\bdelta}{\mbox{\boldmath $\delta$} }
\newcommand{\bepsilon}{\mbox{\boldmath $\epsilon$} }
\newcommand{\bgamma}{\mbox{\boldmath $\gamma$} }
\newcommand{\blambda}{\mbox{\boldmath $\lambda$} }
\newcommand{\bmu}{\mbox{\boldmath $\mu$} }
\newcommand{\bnu}{\mbox{\boldmath $\nu$} }
\newcommand{\bomega}{\mbox{\boldmath $\omega$} }
\newcommand{\bphi}{\mbox{\boldmath $\phi$} }
\newcommand{\bpsi}{\mbox{\boldmath $\psi$} }
\newcommand{\brho}{\mbox{\boldmath $\rho$} }
\newcommand{\bsigma}{\mbox{\boldmath $\sigma$} }
\newcommand{\btau}{\mbox{\boldmath $\tau$} }
\newcommand{\btheta}{\mbox{\boldmath $\theta$} }
\newcommand{\bupsilon}{\mbox{\boldmath $\upsilon$} }
\newcommand{\bxi}{\mbox{\boldmath $\xi$} }
\newcommand{\bzeta}{\mbox{\boldmath $\zeta$} }
\newcommand{\bDelta}{\mbox{\boldmath $\Delta$} }
\newcommand{\bGamma}{\mbox{\boldmath $\Gamma$} }
\newcommand{\bLambda}{\mbox{\boldmath $\Lambda$} }
\newcommand{\bPhi}{\mbox{\boldmath $\Phi$} }
\newcommand{\bSigma}{\mbox{\boldmath $\Sigma$} }
\newcommand{\bTheta}{\mbox{\boldmath $\Theta$} }

\newcommand{\bfa}{\mbox{\bf a} }
\newcommand{\bfb}{\mbox{\bf b} }
\newcommand{\bfc}{\mbox{\bf c} }
\newcommand{\bfd}{\mbox{\bf d} }
\newcommand{\bfe}{\mbox{\bf e} }
\newcommand{\bff}{\mbox{\bf f} }
\newcommand{\bfg}{\mbox{\bf g} }
\newcommand{\bfh}{\mbox{\bf h} }
\newcommand{\bfi}{\mbox{\bf i} }
\newcommand{\bfj}{\mbox{\bf j} }
\newcommand{\bfk}{\mbox{\bf k} }
\newcommand{\bfl}{\mbox{\bf l} }
\newcommand{\bfm}{\mbox{\bf m} }
\newcommand{\bfn}{\mbox{\bf n} }
\newcommand{\bfo}{\mbox{\bf o} }
\newcommand{\bfp}{\mbox{\bf p} }
\newcommand{\bfq}{\mbox{\bf q} }
\newcommand{\bfr}{\mbox{\bf r} }
\newcommand{\bfs}{\mbox{\bf s} }
\newcommand{\bft}{\mbox{\bf t} }
\newcommand{\bfu}{\mbox{\bf u} }
\newcommand{\bfv}{\mbox{\bf v} }
\newcommand{\bfw}{\mbox{\bf w} }
\newcommand{\bfx}{\mbox{\bf x} }
\newcommand{\bfy}{\mbox{\bf y} }
\newcommand{\bfz}{\mbox{\bf z} }
\newcommand{\bfA}{\mbox{\bf A} }
\newcommand{\bfB}{\mbox{\bf B} }
\newcommand{\bfC}{\mbox{\bf C} }
\newcommand{\bfD}{\mbox{\bf D} }
\newcommand{\bfE}{\mbox{\bf E} }
\newcommand{\bfF}{\mbox{\bf F} }
\newcommand{\bfG}{\mbox{\bf G} }
\newcommand{\bfH}{\mbox{\bf H} }
\newcommand{\bfI}{\mbox{\bf I} }
\newcommand{\bfJ}{\mbox{\bf J} }
\newcommand{\bfK}{\mbox{\bf K} }
\newcommand{\bfL}{\mbox{\bf L} }
\newcommand{\bfM}{\mbox{\bf M} }
\newcommand{\bfN}{\mbox{\bf N} }
\newcommand{\bfO}{\mbox{\bf O} }
\newcommand{\bfP}{\mbox{\bf P} }
\newcommand{\bfQ}{\mbox{\bf Q} }
\newcommand{\bfR}{\mbox{\bf R} }
\newcommand{\bfS}{\mbox{\bf S} }
\newcommand{\bfT}{\mbox{\bf T} }
\newcommand{\bfU}{\mbox{\bf U} }
\newcommand{\bfV}{\mbox{\bf V} }
\newcommand{\bfW}{\mbox{\bf W} }
\newcommand{\bfX}{\mbox{\bf X} }
\newcommand{\bfY}{\mbox{\bf Y} }
\newcommand{\bfZ}{\mbox{\bf Z} }

\newcommand{\calA}{{\cal A}}
\newcommand{\calB}{{\cal B}}
\newcommand{\calC}{{\cal C}}
\newcommand{\calD}{{\cal D}}
\newcommand{\calE}{{\cal E}}
\newcommand{\calF}{{\cal F}}
\newcommand{\calG}{{\cal G}}
\newcommand{\calH}{{\cal H}}
\newcommand{\calI}{{\cal I}}
\newcommand{\calJ}{{\cal J}}
\newcommand{\calK}{{\cal K}}
\newcommand{\calL}{{\cal L}}
\newcommand{\calM}{{\cal M}}
\newcommand{\calN}{{\cal N}}
\newcommand{\calO}{{\cal O}}
\newcommand{\calP}{{\cal P}}
\newcommand{\calQ}{{\cal Q}}
\newcommand{\calR}{{\cal R}}
\newcommand{\calS}{{\cal S}}
\newcommand{\calT}{{\cal T}}
\newcommand{\calU}{{\cal U}}
\newcommand{\calV}{{\cal V}}
\newcommand{\calW}{{\cal W}}
\newcommand{\calX}{{\cal X}}
\newcommand{\calY}{{\cal Y}}
\newcommand{\calZ}{{\cal Z}}

\renewcommand{\Hat}{\widehat}
\renewcommand{\Bar}{\overline}
\renewcommand{\Tilde}{\widetilde}

\newcommand{\iid}{\stackrel{iid}{\sim}}
\newcommand{\indep}{\overset{ind}{\sim}}
\newcommand{\argmax}{{\mathop{\rm arg\, max}}}
\newcommand{\argmin}{{\mathop{\rm arg\, min}}}
\newcommand{\Frechet}{ \mbox{Fr$\acute{\mbox{e}}$chet} }
\newcommand{\Matern}{ \mbox{Mat$\acute{\mbox{e}}$rn} }

\providecommand{\argmin}[1]{\underset{{#1}}{\rm \ argmin}} 
\providecommand{\argmax}[1]{\underset{{#1}}{\rm \ argmax}} 

\newcommand{\seteq}{\stackrel{set}{\ =\ }}

\newcommand{\bfig}{\begin{figure}}
\newcommand{\efig}{\end{figure}}
\newcommand{\beqx}{\begin{equation*}}
\newcommand{\eeqx}{\end{equation*}}
\newcommand{\beq}{\begin{equation}}
\newcommand{\eeq}{\end{equation}}
\newcommand{\beqa}{\begin{eqnarray}}
\newcommand{\eeqa}{\end{eqnarray}}
\newcommand{\beqax}{\begin{eqnarray*}}
\newcommand{\eeqax}{\end{eqnarray*}}
\newcommand{\beqn}{\begin{dmath}}
\newcommand{\eeqn}{\end{dmath}}
\newcommand{\beqnx}{\begin{dmath*}}
\newcommand{\eeqnx}{\end{dmath*}}

\let\originalleft\left
\let\originalright\right
\renewcommand{\left}{\mathopen{}\mathclose\bgroup\originalleft}
\renewcommand{\right}{\aftergroup\egroup\originalright}

\providecommand{\itbf}[1]{\textit{\textbf{#1}}} 
\providecommand{\abs}[1]{\left\lvert#1\right\rvert} 
\providecommand{\norm}[1]{\left\lVert#1\right\rVert}

\newcommand{\cond}{\,\left\vert\vphantom{}\right.}
\newcommand{\Cond}{\,\Big\vert\vphantom{}\Big.}
\newcommand{\COND}{\,\Bigg\vert\vphantom{}\Bigg.}

\providecommand{\paren}[1]{\left(#1\right)} 
\providecommand{\Paren}[1]{\Big(#1\Big)}
\providecommand{\PAREN}[1]{\bigg(#1\bigg)} 
\providecommand{\bracket}[1]{\left[ #1 \right]} 
\providecommand{\Bracket}[1]{\Big[ #1 \Big]} 
\providecommand{\BRACKET}[1]{\bigg[ #1 \bigg]} 
\providecommand{\curlybrace}[1]{\left\{ #1 \right\}} 
\providecommand{\Curlybrace}[1]{\Big\{ #1 \Big\}} 
\providecommand{\CURLYBRACE}[1]{\bigg\{ #1 \bigg\}} 

\newcommand{\Bern}{\mbox{{\sf Bern}}}
\newcommand{\Bernoulli}{\mbox{{\sf Bernoulli}}}
\newcommand{\Beta}{\mbox{{\sf Beta}}}
\newcommand{\Bin}{\mbox{{\sf Bin}}}
\newcommand{\Binomial}{\mbox{{\sf Binomial}}}
\newcommand{\DE}{\mbox{{\sf DE}}}
\newcommand{\Exponential}{\mbox{{\sf Exponential}}}
\newcommand{\F}{\mbox{{\sf F}}}
\newcommand{\Gam}{\mbox{{\sf Gamma}}}
\newcommand{\GP}{\mbox{{\sf GP}}}
\newcommand{\GPD}{\mbox{{\sf GPD}}}
\newcommand{\Geom}{\mbox{{\sf Geom}}}
\newcommand{\Geometric}{\mbox{{\sf Geometric}}}
\newcommand{\HyperGeom}{\mbox{{\sf HyperGeom}}}
\newcommand{\HyperGeometric}{\mbox{{\sf HyperGeometric}}}
\newcommand{\InverseGam}{\mbox{{\sf InverseGamma}}}
\newcommand{\InvWish}{\mbox{{\sf InvWish}}}
\newcommand{\MVN}{\mbox{{\sf MVN}}}
\newcommand{\NB}{\mbox{{\sf NB}}}
\newcommand{\NegBin}{\mbox{{\sf NegBin}}}
\newcommand{\NegativeBinomial}{\mbox{{\sf NegativeBinomial}}}
\newcommand{\Normal}{\mbox{{\sf Normal}}}
\newcommand{\Pois}{\mbox{{\sf Pois}}}
\newcommand{\Poisson}{\mbox{{\sf Poisson}}}
\newcommand{\Unif}{\mbox{{\sf Unif}}}
\newcommand{\Uniform}{\mbox{{\sf Uniform}}}
\newcommand{\Weibull}{\mbox{{\sf Weibull}}}

\renewcommand{\P}{{\sf P}}
\newcommand{\Prob}{{\sf Prob}}
\newcommand{\median}{{\mathop{\rm median}}}
\newcommand{\E}{\mathsf{E}}
\newcommand{\V}{\mathsf{V}}
\newcommand{\VAR}{\mathsf{VAR}}
\newcommand{\COV}{\mathsf{COV}}

\newcommand{\Ind}{\mathds{1}}
\newcommand{\zerovect}{\mbox{\bf 0}}
\newcommand{\onesvect}{\mbox{\bf 1}}
\providecommand{\real}[1]{\mathbb{#1}}
\newcommand{\Real}{\mathbb{R}}
\newcommand{\ppd}{\mathcal{P}}
\DeclareMathOperator{\logit}{logit}
\DeclareMathOperator{\expit}{expit}
\DeclareMathOperator{\dint}{\displaystyle\int}
\DeclareMathOperator{\dsum}{\displaystyle\sum}

%============================================
%My New Commands
%============================================
%============================================


\newcommand{\bitemize}{\begin{itemize}\setlength{\itemsep}{1pt}\setlength{\parskip}{1pt}}
\newcommand{\eitemize}{\end{itemize}}
\newcommand{\benum}{\begin{enumerate}\setlength{\itemsep}{1pt}\setlength{\parskip}{1pt}}
\newcommand{\eenum}{\end{enumerate}}

\usepackage{fancyhdr}
\pagestyle{fancy}

%\lhead{\footnotesize \parbox{11cm}{Custom left-head-note} }
\cfoot{}
\lfoot{\footnotesize \parbox{11cm}{}}
\rfoot{\footnotesize Page \thepage\ }
%\rfoot{\footnotesize Page \thepage\ of \pageref{LastPage}}
%\renewcommand\headheight{24pt}
\renewcommand\footrulewidth{0.4pt}


\usepackage[colorlinks=false,
          %  pdfborder={0 0 0},
            ]{hyperref}


%  My Packages and Commands %
%======================================================%
%-----------------------------------------------------------------------------------------------%
%****************************************************************************%
%++++++++++++++++++++++++++++++++++++++++++++++++++++++%
%~~~~~~~~~~~~~~~~~~~~~~~~~~~~~~~~~~~~~~~~~~~~~~~~~~~~~~%
%______________________________________________________%






%============================================
%============================================
% PREAMBLE
%============================================
%============================================

\title[Modeling dependencies in multivariate seabird distributions]{Modeling dependencies in multivariate seabird distributions: linking spatial dependence graph models to centrality measures}
\author[Author 1 {\it et al.}]{Author 1}
\address{Affiliation,
City,
         Country.}
\email{Author@emailaddress.com}
\author{Author 2}
\address{Affiliation,
         City,
         Country.}
         




%============================================
%============================================
\IfFileExists{upquote.sty}{\usepackage{upquote}}{}
\begin{document}%\linenumbers
%============================================
%============================================



%============================================
\begin{abstract}
  Graphical modeling of seabird distributions.
\end{abstract}

\keywords{Graphical modeling, Species importance, Marked point process, Network centrality measures, Seabird distribution}
%============================================



%============================================
\section{Introduction}
%============================================

\begin{itemize}
\item We estimate the conditional spatial dependence structure between different seabird species
\item edges express the partial (pairwise) interrelation between two species conditional on al remaining species
\item we omit to implement a formal test statistic as we assume that dependence might vary between species
\item we set a threshold indicating weak/ intermediate partial effects
\item we combine spatial dependence graph model with importance tools of social network analysis
\item info can provide important insides for species conservation
\item eg which are the most important species in the overall graph
\end{itemize}

%============================================
\section{Methods}
%============================================

%============================================
\section{Data}
%============================================

species: 
Herring gull $(n=3978)$, greater shearwater $(n=3732)$,
northern gannet $(n=3451)$,great black-backed gull $(n=3319)$,Wilson's storm-petrel $(n=2746)$,common loon $(n=1900)$,northern fulmar, $(n=1688)$,red-throated loon $(n=1400)$,Cory's shearwater $(n=1138)$, Leach's storm-petrel $(n=1005)$,unidentified gull $(n=984)$,common tern $(n=982)$,black-legged kittiwake $(n=971)$,dovekie $(n=962)$,razorbill $(n=908)$, sooty shearwater $(n=871)$, laughing gull $(nl=856)$, long-tailed duck $(n=851)$,unidentified loon $(n=754)$, Bonaparte's gull $(n=708)$,unidentified scoter $(n=703)$, surf scoter $(n=678)$,
black scoter $(n=669)$, unidentified tern $(n=619)$,
unidentified alcid $(n=555)$, unidentified large gull $(n=553)$, unidentified bird $(n=543)$, ring-billed gull $(n=530)$, common eider $(n=528)$, white-winged scoter $(n=522)$, atlantic puffin $(n=403)$, red phalarope $(n=394)$, unidentified storm-petrel $(n=379)$, manx shearwater $(n=374)$, unidentified shearwater $(n=364)$, double-crested cormorant $(n=354)$, unidentified large alcid (razorbill or murre) $(n=330)$, unidentified small gull $(n=313)$,ubbg $(n=299)$,unidentified phalarope $(n=292)$,pomarine jaeger $(n=287)$,red-breasted merganser $(n=220)$,red-necked phalarope $(n=207)$,royal tern $(n=207)$,dark scoter $(n=199)$,unidentified small tern $(n=184)$,south polar skua $(n=169)$,bufflehead $(n=159)$,parasitic jaeger $(n=154)$,common murre $(n=153)$,unidentified diving/sea duck $(n=121)$,Forster's tern $(n=119)$,great skua $(n=117)$,unidentified merganser $(n=117)$,roseate tern $(n=115)$,arctic tern $(n=103)$,barn swallow $(n=99)$,unidentified cormorant $(n=99)$,lesser black-backed gull $(n=97)$,unidentified passerine $(n=97)$,brown pelican $(n=92)$,unidentified large tern $(n=91)$,unidentified skua $(n=89)$,unidentified shorebird $(n=87)$,Audubon's shearwater $(n=86)$,thick-billed murre $(n=83)$,common goldeneye $(n=82)$,unidentified jaeger $(n=78)$,black guillemot $(n=75)$,unidentified murre $(n=73)$,horned grebe $(n=65)$,unidentified grebe $(n=63)$,least tern $(n=61)$,unidentified scaup $(n=52)$,
red-necked grebe $(n=49)$,American black duck $(n=47)$,unidentified goldeneye $(n=34)$,band-rumped storm-petrel $(n=23)$ and, finally,
black-capped petrel $(n=3)$.

While some marine bird species such as various types of gulls have been sighted at numerous locations, other species occurred only very rarely. At the same time, as we defined a minimum of at least $50$ marine birds per species as inclusion criteria, this limited number of locations equivalently implies that certain marine bird species appeared in groupings rather than as isolated birds or occurred only in a geographically strictly-limited habitat. An example of such only rarely observed species are black-capped petrels whose $50$ counted sighting have only been recorded at $n=3$ different locations. This reflects our expectations, as black-capped petrels have been classified as endangered species by the IUCN Red List of Threatened Species. Discovering the relevant subset of species whose spatial occurrence is linked to the spatial pattern of any endangered species  by means of a SDGM might provide new insights into multivariate interdependencies.  This new insights might provide important knowledge for the conservation of endangered species and might assist to understand such phenomena from a global perspective on different natural environments.    

%============================================
\section{Results}
============================================

$\star$ in general: important for species conservation: e.g. black-capped petrels only linked to unidentified goldeneyes, both listed as threatened species by IUCN. 


$\star$ most important species by means of the degree centrality are unidentified alcids, Arctic terns, barn swallows, red phalaropes, dovekies, buffleheads, brown pelicans, unidentified mergansers and great black-backed gulls.


$\star$ these species are most often linked to alternative species (no. adjacent species)


$\star$ buffleheads, royal terns, unidentified alcids, Arctic terns, common murres, barn swallows, greater shearwaters, manx shearwaters, Bonaparte's gulls and Northern gannets are the most important species by means of the betweenness centrality.


$\star$ these species are most often intermediate species, that is most links from a to b path through these species


\end{document}
